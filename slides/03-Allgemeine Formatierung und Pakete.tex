\input{kursheader}

\coursesetup{
title=Formatieren und Definieren,
subtitle=Teaching the Lion,
number=03
}
\usepackage[xspace]{ellipsis}
\MakeShortVerb{|}

\begin{document}
\section{Tips und Tricks für die Arbeit mit \TeX works}
\begin{frame}[fragile]{\TeX works}{Nützliche Feinheiten}
• Einstellung der möglichen Tools: \\%
\texttt{Edit} ↦ \texttt{Preferences} ↦ \texttt{Typesetting}
\pause
• Autovervollständigung: Viele Befehle sind bekannt und können vervollständigt werden:\\
|\doc <tab>| ⇒ |\documentclass{}|\\
|bit <tab>| ⇒ |\begin{itemize} \end{itemize}|
• bei Mehrdeutigkeit: mehrmals |<tab>| drücken
• Einstellbar über Datei |tw-latex.txt|, meist im Nutzerordner unter |./TeXworks/completion/tw-latex.txt|
\•
\end{frame}

\section{Der Satzspiegel}
\begin{frame}{Formatierung: Makrotypographie}
• „globale“ Typographie:
• Satzspiegel
• Gestaltung von Kopf- und Fußzeilen 
• Wahl der Schriften
• Formatierung von Abständen u.\,ä.
• Inhaltsverzeichnis, Indizes, Verzeichnisse …
\• 
\end{frame}


\begin{frame}{Der Satzspiegel}
Bezeichnet alles, was auf der Seite zum Bedrucken genutzt wird:
• Verhältnis von Rändern zu Textbreiten
• ein- oder zweiseitiger Satz
• ein- oder mehrspaltiger Satz
• Kopf- und Fußzeilen
• Satzspiegel abhängig von Schriftgröße und Laufweite, für ein harmonisches Bild\pause
\•
Leider ist der Satzspiegel meist (vom Herausgeber, Professor, …) vorgegeben, oft mit typographisch fragwürdigen Einstellungen – zu kleine oder unpassende Ränder, unharmonische Seitenaufteilung etc.
\end{frame}

\begin{frame}{Satzspiegel mit KOMA}
Als Alternative zu den Standardklassen bietet das KOMA-Skript (Hauptautor Markus Kohm):
• optimale Satzspiegelkonstruktion
• mit dem speziellen Paket |typearea|
• automatische Berechnung (Paket laden, fertig)
• anpassbar für Spezialfälle (z.\,B. mittelalterliche Satzspiegelkonstruktion)\pause
\• 
Falls das alles nicht gefällt: freie Anpassung aller Dimensionen mittels |geometry| (unabhängig von KOMA)
\end{frame}

\begin{frame}[fragile]{Das Paket |geometry|}
• erlaubt manuelle Einstellung des Satzspiegels
• alle Parameter werden mittels Paketoption angegeben:\\%
|\usepackage[top=2cm,bottom=5cm]{geometry}|
• \emph{oder} mittels |\geometry{top=2cm,bottom=5cm}|
• mittels |\newgeometry| kann der Satzspiegel geändert werden
• |\restoregeometry| stellt vorherige Einstellungen wieder her
• |\savegeometry{}| und |\loadgeometry{}| erlauben Speichern und Laden von Einstellungen
\•
\end{frame}

\begin{frame}[fragile]{Das Paket |geometry|}
\begin{block}{Mögliche Optionen (kleine Auswahl):}
\begin{verbatim}
paperheight, paperwidth
left, right, inner, outer, hmargin
top, bottom, vmargin
bindingoffset, textwidth, textheight
twoside, twocolumn, columnsep
marginparsep, footnotesep, headsep, footsep, nofoot, nohead
hoffset, voffset, offset
\end{verbatim}
\vdots 
⇒ siehe |texdoc geometry| für ausführliche Beschreibung und Dokumentation
\end{block}
\end{frame}

\section{Kopf- und Fußzeilen}
\begin{frame}[fragile]{Kopf- und Fußzeilen}
• Enthalten wichtige Informationen über das Dokument:
• Lebende Kolumnentitel (Kapitel-/ Abschnittsüberschriftend)
• Seitenzahlen
• evtl. kleine Verzierungen, Schnörkel, …
\•
\pause

• Verschiedene Pakete erlauben Anpassungen
• Auswahl des Seitenstils mittels\\%
|\pagestyle{Seitenstil}| (für alle Seiten) oder\\%
 |\thispagestyle{Seitenstil}| (nur diese Seite)
• Voreinstellungen im \LaTeX-Kernel:\\%
 |empty|, |plain|, |headings|
\•
\end{frame}

\begin{frame}[fragile]{Seitenstil mit KOMA: |scrpage2|}
\begin{lstlisting}
\usepackage[automark]{scrpage2}
\pagestyle{scrheadings} %% entweder scrheadings oder scrplain
\pagestyle{scrplain} %%    muss aktiviert werden
\lehead[scrplain-links-gerade]{scrheadings-links-gerade}
\cohead[scrplain-zentriert-ungerade]{scrheadings-z-u}
\rehead[]{}
\cefoot[]{}
\ohead[]{\pagemark}
\ihead{\headmark}
\cfoot{}
\end{lstlisting}
⇒ siehe |texdoc scrguide|
\end{frame}

\begin{frame}[fragile]{Seitenstil anpassen mit |fancyhdr|}
Laden des Pakets und Einstellungen im Dokumentenkopf:
\begin{lstlisting}
\usepackage{fancyhdr}
\pagestyle{fancy} %% ohne das funktioniert nichts
\end{lstlisting}
\begin{columns}
\begin{column}{.4\textwidth}
Einseitiger Satz:
\begin{lstlisting}
\lhead{}    \lfoot{}
\chead{}    \cfoot{}
\rhead{}    \rfoot{}
\end{lstlisting}
\end{column}
\pause
\begin{column}{.4\textwidth}
Zweiseitiger Satz (Odd, Even)
\begin{lstlisting}
\fancyhead[LO]{}
\fancyhead[RO,LE]{}
\fancyhead[CE]{}
\fancyfoot[LO]{}
\fancyfoot[RO,LE]{}
\fancyfoot[CE]{}
\end{lstlisting}
\end{column}
\end{columns}
\end{frame}

\section{Schriftgrößen}
\begin{frame}[fragile]{Schriftgrößen}
\begin{block}{Dokumentoption}
Fast alle Klassen bieten Optionen für die Standardschriftgröße:\\
|\documentclass[10pt]{scrartcl}|
\end{block}

\begin{block}{abgeleitete Größen}
Davon abgeleitet sind alle anderen Größen (|\tiny|, |\small|, … ), speziell |\footnotesize| und |\scriptsize|.\\
⇒ vordefinierte Schriftgrößen erlauben konsistentes Layout.
\end{block}
\pause
Wer wirklich \emph{genau weiß}, was er tut:
\begin{verbatim}
\fontsize{Größe}{Durchschuss}\selectfont
\fontsize{10}{12}\selectfont
\end{verbatim}
\end{frame}

\section{Typographie: Leerzeichen, Striche}
\begin{frame}[fragile]{Typographisch korrekte Strichlängen}
\begin{block}{Leerzeichen und Striche}
Je nach Zweck, Position und Bedeutung werden verschiedene horizontale Striche und Leerräume verwendet.
\end{block}
\pause
• Grammatik und Lesbarkeit bestimmen die nötige Strichlänge
• Zu unterscheiden:
• Viertelgeviertstrich oder Bindestrich: |-|
• Halbgeviertstrich oder Gedankenstrich: |--| –
• Geviertstrich oder englischer Gedankenstrich: |---| —
• Minuszeichen: |$-$| $-$
• Im Vergleich:\huge%
\\ |- -- --- $- +$| {\normalsize (Eingabe, falls keine Unicodeeingabe)}%
\\ - – — $- +$ {\normalsize (Ausgabe)}%
\•
\end{frame}

\begin{frame}[fragile]{Typographisch korrekte Strichlängen}
\begin{block}{Verwendung}
Vorder- und Rückseite \dots\\
Vorderseite – oder auch Rückseite – \dots\\
frontmatter—or backmatter— \dots\\
Längen/Zeitangaben: Düsseldorf\,–\,Koblenz; 18–20\,Uhr\\%
Unterschiedliche Empfehlungen für das Setzen von Spatia (kleine Leerzeichen, siehe unten)
\end{block}
\end{frame}

\begin{frame}[fragile]{Leerräume}
• normales Leerzeichen: \verb*| |, erzwungenes Leerzeichen: \verb*|\ |, nichttrennbares Leerzeichen: |~|
• schmales Leerzeichen (Spatium): |\,|
• negativer Abstand: |\!|\\%
 vergleiche normal, schmal, negativ: z. B., z.\,B., z.\!B.
• großer Abstand (Geviert) |\emspace|
• beliebiger Platz: |\hspace(*){2em}|
• Leerräume im Mathemodus:%
\\ (werden normal automatisch korrekt gesetzt)%
\\ |$a b = a\, b = a \! b$|
\\ $a b = a\, b = a \! b$%
\pause
• Explizites Ändern des Abstandes (kerning):%
\\ |a\kern-0.1em b| ⇒ a\kern-0.1em b \notiz{Mögliche Maße an der Tafel}%
\\ |a\kern-0.5em b| ⇒ a\kern-0.5em b 
\• 
\end{frame}

\begin{frame}[fragile]{Auslassungen}
Auslassungspunkte sollten gesperrt gesetzt werden:
\\... (falsch) \\\ldots (richtig)
\\ Wenn ganze Satzteile \ldots %% FIXME: Dank otf tut das nicht mehr. WTF?
\\ ausgelassen werden \ldots\ setzt man Leerzeichen.
\\ Wenn Wortteile ausgelassen werden, kommt kein Leer\dots\ davor.
\\[2em]
Befehle |\dots| oder |\ldots| sorgen für richtige Abstände.\\
Paket |ellipsis| korrigiert den Abstand vor und nach Ellipsen.\\
\alert{Achtung, fontspec:} bei Verwendung von \textsf{fontspec} \emph{muss} das \textsf{ellipsis}-Paket (\emph{nach} \textsf{fontspec}) geladen werden, da sonst falsche Abstände resultieren.\pause
\begin{block}{Pro-Tip}
Ausnutzen aktiver Zeichen (nur mit Vorsicht!)
\begin{verbatim}
\catcode`…=13
\let…\dots
\end{verbatim}
\end{block}
\end{frame}

\begin{frame}[fragile]{Probleme mit |ellipsis|}
|\usepackage{ellipsis}| führt zu inkonsistenten Abständen nach |\dots|:
\begin{lstlisting}
\LaTeX ist \dots manchmal kompliziert \dots
\end{lstlisting}
\LaTeX ist \dots{}manchmal kompliziert \dots
\\[1em]\pause
|\usepackage[xspace]{ellipsis}| korrigiert dies:
\begin{lstlisting}
\LaTeX\ ist \dots manchmal kompliziert \dots
\end{lstlisting}
\LaTeX\ ist \dots\ manchmal kompliziert \dots
\end{frame}

\section{Dokumentation lesen und verstehen}
\begin{frame}[fragile]{Welche Befehle möglich?}
• Woher weiß man, welche Befehle |ellipsis| oder |geometry| bieten?
• Welche Einstellungen haben die KOMA-Klassen?
• Wie bekommt man allgemein Informationen über Pakete\,/\,Klassen?
\•
\end{frame}

\begin{frame}[fragile]{Dokumentationen mittels texdoc}
• |texdoc| durchsucht die \LaTeX-Ordner
• liefert direkt die Dokumentation des gesuchten Paketes:
• |texdoc amsmath| öffnet |amsmath.pdf|
• |texdoc -l amsmath| listet alle Ergebnisse auf
• |texdoc -s amsmath| liefert Ergebnisse aus erweiterter Suche
• |texdoc --help| bietet weitere Informationen%
\•
⇒ Vorteile der Kommandozeile\\[2em]
|texdoctk| bietet graphische Oberfläche mit Auswahlmenüs – gut zum „Blättern“, wenn man nicht weiß, was man sucht.
\end{frame}

\section{Befehle und Umgebungen definieren}
\begin{frame}[fragile]{Makros}
• Abkürzungen, die das Leben erleichtern
• \LaTeX\ definiert große Zahl von Makros vor
• Klassen und Pakete erweitern die Möglichkeiten immens
• für den Eigenbedarf kann der Nutzer selbst neue Makros definieren
• Makros nehmen Argumente an, die mandatorisch oder optional sein können.
• Argumente sind einzelne Zeichen (genauer, Token) oder Gruppen |{abc}| in geschweiften Klammern.
• Optionale Argumente meist in eckigen Klammern, aber Paketautoren sind kreativ: runde, spitze Klammern u.\,ä. kommt vor.
\•
\end{frame}

\begin{frame}[fragile]{Neue Befehle mit |xparse|}
• \alert{\LaTeX3} bietet durch das |xparse|-Paket hervorragende Syntax für neue Befehle
• für \emph{Dokumentenebene} geeignet (also Paketautoren und Dokumentautoren)
• |\NewDocumentCommand\cmd{args}{def}|
• Makro ohne Argumente:\\ 
|\NewDocumentCommand\cmd{}{Code}| setzt den angegbenen Code.\pause
• ein Argument:\\
|\NewDocumentCommand\cmd{m}{Text: #1}|
• |m| für |mandatory| – muss vorhanden sein.
• Argument kann mittels |#1| verwendet werden.
\•
\end{frame}

\begin{frame}[fragile]{Argumenttypen in |xparse| (Auswahl)}
\begin{description}
\item[m] mandatory – muss gegeben werden, in geschweiften Klammern
\item[o] optional – kann vorhanden sein oder nicht, in eckigen Klammern
\item[O\{default\}] dito, mit Voreinstellung
\item[d<>] delimiter – ebenfalls optional, mit den angegebenen Begrenzungen |\cmd<a>|
\item[D()\{default\}] dito, mit Voreinstellung, und Begrenzenungen |()|
\item[g/G] Spezialfall für |d/D{}|
\item[t\{X\}] testet, ob das Token |X| an der angegeben Stelle steht. (boolean)
\item[s] Sternversion, Spezialfall von |t{*}|
\item[u\{X\}] liest ein Argument bis (until) dem angegebenen Token ein, mandatorisch
\item[l] liest bis zur ersten öffnenden |{|, mandatorisch
\end{description}
\end{frame}

\begin{frame}[fragile]{Neue Befehle „aus dem Antiquariat“}
• \TeX-Basisbefehl zum definieren neuer Makros: |\def|
• \LaTeX\ definiert flexiblere Definitionen mit speziellen Absicherungen
• \LaTeXe: |\newcommand|, |\renewcommand| und |\declarecommand|
\•
|\newcommand[args]\name{definition}|\\
|\newcommand[opt][2]\mycommand{#1, #2}|\\
|\mycommand{ab}{cd}| → ab, cd
\end{frame}

\begin{frame}[fragile]{Leerzeichen in \TeX}
\parbox{.12\textwidth}{\alert{Achtung:}\\ \includegraphics[width=1.4cm]{lion_colour}}
\TeX\ „frisst“ gerne Leerzeichen – vor allem nach Befehlen:\\
|\wasser ist nass| ⇒ H$_2$Oist nass.

\pause
•<+-> \TeX\ liest Befehle vom |\ | bis zum ersten nicht-Buchstaben%
\\ (Zahl, Klammer, Leerzeichen, Punkt, \dots)
\\ \verb|\LaTeX   ist  manchmal   umständlich|
•<+-> Befehle im Text immer mit |\| oder |{}| beenden:
•<+-> \verb|\LaTeX\ ist manchmal umständlich.|
\•
\begin{block}{}\centerline{
\only<2-3>{\LaTeX ist manchmal umständlich}
\only<4->{\LaTeX\ ist manchmal umständlich}}
\end{block}
\pause
⇒ um konsistent zu werden, ignoriert \LaTeX3 auf Paketebene \emph{alle} Leerzeichen!
\end{frame}

\begin{frame}[fragile]{xspace}
• Bei selbstdefinierten Befehlen: |\xspace| aus |xspace|
• |\NewDocumentCommand\wasser{}{H₂O\xspace}|%
• |\xspace| fügt Leerzeichen da ein, wo es sinnvoll ist, lässt es aber weg, wo es nicht hingehört.\\
⇒ ganz gewöhnliches Schreiben ist möglich.\pause
•[⇒] |\wasser ist nass. \wasser, gefroren, ist nicht nass.|%
\NewDocumentCommand\wasser{}{H₂O\xspace}%
•[⇒] \wasser ist nass. \wasser, gefroren, ist nicht nass.\pause
• Für Spezialfälle (besondere Satzzeichen, Sonderzeichen, ?...):\\%
 Anpassung möglich mittels |\xspaceaddexceptions{|ℵ|}|%
\xspaceaddexceptions{ℵ}%
\\ ⇒ |\wasser |ℵ ⇒  \wasser ℵ
\•
\end{frame}

\begin{frame}[fragile]{Low Level: def}
• \TeX\ kennt nur den primitiven Befehl |\def|
• \alert{low level ⇒ Vorsicht!}
• |xparse| definiert |\NewDocumentCommand| aus guten Gründen!\\ (schützt vor Überschreiben)
• für spezielle Ansprüche kann |\def| aber nützlich sein:
\•

\begin{lstlisting}
\def\dosomething#1#2xyz#3{some code}
\end{lstlisting}
(mit |xparse| auch möglich, aber längere Definition)
\end{frame}

\end{document}